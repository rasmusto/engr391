\documentclass[8pt, letter]{extarticle}
\usepackage[margin=1cm]{geometry}
\usepackage{multicol}
\usepackage{enumitem}

\usepackage{setspace}
\singlespacing
%\onehalfspacing
%\doublespacing
%\setstretch{1.1}


\begin{document}
\small

\begin{multicols}{2}

    \section{Project Management}
    \begin{itemize}
        \item What is a ``project''? - Project: A complex, non-routine, one-time effort limited by time, budget, resources, and performance specifications designed to meet customer needs.
        \item Projects cost money and bring benefit
    \end{itemize}

    \subsection{Comparison of Routine Work with Projects}
    \begin{itemize}
        \item Routine, Repetitive Work
        \item Programs versus Projects
        \item Program:  A series of coordinated, related, multiple projects that continue over an extended time and are intended to achieve a goal.
        \item A higher level group of projects targeted at a common goal.
        \item Example:
            \begin{itemize}
                \item Project: completion of a required course in project management.
                \item Program: completion of all courses required for a business major. 
                \item Projects have a beginning and an end
                \item Projects start with a defined objective
            \end{itemize}
        \item Projects end:
            \begin{itemize}
                \item When the objective is met
                \item When the project results are not meeting organizational needs
            \end{itemize}

        \item Engineering concerns solving problems through projects that cost money and bring benefit.
        \item What are its major characteristics?
            \begin{itemize}
                \item Has an established objective.
                \item Has a defined life span with a beginning and an end.
                \item Requires across-the-organizational participation.
                \item Involves doing something never been done before.
                \item Has specific time, cost, and performance requirements.
            \end{itemize}
    \end{itemize}

    \subsection{Project life cycle}
    \begin{itemize}
        \item Defining stage – objects established, teams are formed, responsibilities are assigned. 
        \item Planning stage – what project will entail, when it will be scheduled, whom it will benefit, what quality level should be maintained, and what the budget will be. 
        \item Executing stage – most work takes place. Time, cost, measures. 
        \item Delivering stage: Two activities : delivering the project product to the customer and redeploying project resources.
    \end{itemize}

    \subsection{The strategic management process}
    \begin{enumerate}
        \item Review and define the Organizational mission. 
        \item Set long-range goals and objectives. 
        \item Analyze and formulate strategies to reach objectives. 
        \item Implement strategies through projects.
    \end{enumerate}

    \subsection{Project Portfolios and types of projects}
    \begin{enumerate}
        \item Compliance - absolutely must do, oil spill
        \item Operational - not routine, but required to support current operations
        \item Strategic - advances company mission, improves market share, etc.
    \end{enumerate}

    \subsection{Project Selection models}
    \begin{tabular}{ l l l l l }
        Criteria    & Urgency & Strategic fit & ROI of 18\%+ & Weighted Total     \\
        Weight      & 2.0   & 3.0   & 3.0   & --    \\
        Project 1   & 1     & 8     & 2     & 32    \\
        Project 2   & 3     & 3     & 4     & 27    \\
        Project n   & x     & x     & x     & x     \\
    \end{tabular}
    \subsection{Defining the project}
    \begin{itemize}
        \item Project Scope - cost, deliverables, time, features, etc.
        \item Project Objective statement - does it align with company mission?
        \item Project scope checklist
            \begin{itemize}
                \item Project objective – Define overall objective of customer's need. 
                \item Deliverable s – Expected output over the life of the project. 
                \item Milestones – Schedule shows only major segments of work; it represents first, rough cut estimates of time, costs, and resources. 
                \item Technical requirements – Ensure proper performance. 
                \item Limits and exclusions – Limits of scope should be defined. Review with customer
            \end{itemize}
        \item Importance of project scope
        \item Scope creep - project becomes more than it was intended to be
        \item Project Priority Matrix - apply numbers to proj characteristics
            \\
            \begin{tabular}{ l c c c }
                --&         Time&Performance& Cost  \\
                Constrain   &   O   &       &       \\
                Enhance     &       &   O   &       \\
                Accept      &       &       &   O   \\
            \end{tabular}
        \item Managing the trade-offs
            \begin{itemize}
                \item Scope, Time, Cost all affect quality and are trade-offs
            \end{itemize}
        \item Work Breakdown Structure
            \begin{itemize}
                \item hierarchical organization of all sub-projects
                \item doesn't take time into account
            \end{itemize}
        \item Work Packages—what, who, how long, monetary metrics
        \item Linkages between tasks
        \item Gantt Chart
            \begin{enumerate}
                \item Networks flow typically from left to right.
                \item Anactivity cannot begin until all preceding connected activities have been completed.
                \item Arrows on networks indicate precendence and flow. Arrows can cross over each other.
                \item Each activity should have a unique identification number.
                \item Anactivity identification number must be larger than that of any activities that precede it.
                \item Looping is not allowed (in other words, recycling through a set of activities cannot take place).
            \end{enumerate}
        \item Communication Plans
            \begin{itemize}
                \item Stakeholder analysis – Identify the target groups. 
                \item Information needs – what information is stakeholders to contribute to project's progress. 
                \item Sources of information – what information needs are identified, the next step is to determine the sources of information. 
                \item Dissemination modes – methods of information travel. 
                \item Responsibility and timing – Determine who will send out the information.
            \end{itemize}
    \end{itemize}

    \subsection{Estimates}
    \begin{itemize}
        \item Factors affecting estimates
        \item Top Down: 
            \begin{itemize}
                \item Consensus - Uses the pooled experiences of senior and middle managers to estimate.
                \item Ratio - \$ per foot of construction project
                \item Apportionment - extension of ratio, used when there is known historical data.
                \item Function Point - Major parameters, \#I/O, interfaces, data, inquiries.
                \item Learning Curve - Time to perform a task improves with repetition.  Each time the output quantity doubles, the unit labor hours are reduced at as constant rate.
            \end{itemize}
        \item Bottom Up:
            \begin{itemize}
                \item Template - Past projects are used as a starting point.
                \item Parametric - Similar to ratio, but from the ground up.
                \item Detailed - Use the WBS and have people responsible for each section make the estimates.
            \end{itemize}
        \item Phase Estimating (a hybrid) 
        \item Refining Estimates
            \begin{itemize}
                \item Interaction costs are hidden in estimates.  Tasks are rarely completed in a vacuum.
                \item Normal conditions do not apply. Availability of resources may change.
                \item Things go wrong on projects. Design flaws, etc.
                \item Changes in project scope and plans.
            \end{itemize}
    \end{itemize}

    \subsection{Project Plan Development}
    \begin{itemize}
        \item Network Diagram
            \begin{itemize}
                \item Forward Pass
                \item Backward Pass
                \item Schedule
                \item Critical Path
                \item Slack
            \end{itemize}
            \begin{tabular}{ l l l }
                Early Start & ID            & Early Finish  \\
                Slack       & Description   & Slack         \\
                Late Start  & Duration      & Late Finish   \\
            \end{tabular}
    \end{itemize}

    \subsection{Project Managers}
    \begin{itemize}
        \item Manages temporary, non-repetitive activities and frequently acts independently of the formal organization.
        \item Marshals resources for the project.
        \item Is linked directly to the customer interface.
        \item Provides direction, coordination, and integration to the project team.
        \item Is responsible for performance and success of the project.
        \item Must induce the right people at the right time to address the right issues and make the right decisions.
    \end{itemize}

    \subsection{Managing Risk}
    \begin{itemize}
        \item Risk Identification
        \item Events, not outcomes
        \item Use multiple methods
        \item Risk Assessment
        \item FMEA - is a procedure in product development and operations management for analysis of potential failure modes within a system for classification by the severity and likelihood of the failures.
            \begin{itemize}
                \item Failure - "The LOSS of an intended function of a device under stated conditions."
                \item Failure mode - "The manner by which a failure is observed; it generally describes the way the failure occurs."
                \item Failure effect - Immediate consequences of a failure on operation, function or functionality, or status of some item
                \item Indenture levels - An identifier for item complexity. Complexity increases as levels are closer to one.
                \item Local effect - The Failure effect as it applies to the item under analysis.
                \item Next higher level effect - The Failure effect as it applies at the next higher indenture level.
                \item End effect - The failure effect at the highest indenture level or total system.
                \item Failure cause - Defects in design, process, quality, or part application, which are the underlying cause of the failure or which initiate a process which leads to failure.
                \item Severity - "The consequences of a failure mode. Severity considers the worst potential consequence of a failure, determined by the degree of injury, property damage, or system damage that could ultimately occur." 
            \end{itemize}

        \item Scenario Analysis - process of analyzing possible future events by considering alternative possible outcomes (scenarios). 
            \begin{itemize}
                \item does not try to show one exact picture of the future
                \item it presents consciously several alternative future developments
            \end{itemize}
        \item Probability Analysis - analysis using the probability of different scenarios
        \item Risk Response Planning - determine what to do if a risk leads to some consequence
            \begin{itemize}
                \item Retain - deal with it, contingency plan
                \item Mitigate - work to reduce likelihood/severity
                \item Transfer - pay someone to deal with it
                \item Share - spread responsibility among multiple parties
                \item Avoid - change project plan to avoid risk
            \end{itemize}
        \item Contingency Plans
        \item Risk Management and Change Control
    \end{itemize}

    \begin{itemize}
        \item Resource Allocation \& Management
        \item Types of Project Constraints
            \begin{itemize}
                \item Technical/Logic - constraints related to the networked sequence in which project activities must occur
                \item Resource - The absence, shortage, or unique interrelationship and interaction characteristics of resources that require a particular sequencing of project activities
                \item Physical - not enough space to work, project not feasible
            \end{itemize}
        \item Time Constrained or Resource Constrained
        \item Resource Leveling - involves attempting to even out demands on resources by using slack (delaying noncritical activities) to manage resource utilization
        \item Types of Resource Constraints
            \begin{itemize}
                \item People - workers, managers, engineers, techs, etc.
                \item Materials - concrete, pcbs, wood
                \item Equipment - oscilloscopes, backhoes, trucks
                \item Working Capital - money
            \end{itemize}
        \item Effects on logical plan
        \item Heuristics for Resource Leveling
            \begin{enumerate}
                \item Minimum slack
                \item Shortest duration
                \item Lowest activity identification number
            \end{enumerate}
    \end{itemize}

    \subsection{Project Execution, Management \& Control}
    \begin{itemize}
        \item WHO needs to know?
        \item WHAT do they need to know?
        \item WHEN do they need to know?
    \end{itemize}

    \subsection{Change Management}
    \begin{itemize}
        \item How to track progress
        \item Baseline Schedule
        \item Baseline Budget
        \item Basic concepts of Earned Value
    \end{itemize}

    \subsection{How to track progress}
    \begin{itemize}
        \item Baseline Schedule
        \item Baseline Budget
        \item Basic concepts of earned value
    \end{itemize}
\end{multicols}
\end{document}
