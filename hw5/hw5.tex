\documentclass[12pt, letter]{extarticle}
\usepackage[margin=1cm]{geometry}
\usepackage{multicol}
\usepackage{enumitem}

\usepackage{setspace}
\singlespacing


\begin{document}

\section{Ch. 8: RQ 2}
How does resource scheduling reduce flexibility in managing projects?

If you have to worry about when you are allowed to use a resource, you will be less productive.
Also, if you finish early, you will not be allowed access to a specific resource until the scheduled time.

\section{Ch. 8: RQ 6}
Why is it critical to develop a time-phased baseline?

Time-phased baselines make up for poor representation of project progress found in other models.
Other project progress testing will judge progress based solely on money spent, but this can be a poor indicator of the progress that has actually been made.

\section{Conveyor Belt Project Part 3a}
\section{Conveyor Belt Project Part 3b}

\section{Ch. 13: RQ 1}
How does a Tracking Gantt chart help communicate project progress?

A Tracking Gantt chart helps to communicate project progress by automatically updating the Gantt chart to reflect current activity progress.  It is able to visually show this, thus providing a good reference for judging project progress.

\section{Ch. 13: RQ 6}
Why is it important for project managers to resist changes to the project baseline?  Under what conditions would a project manager make changes to a baseline? When would a project manager not allow changes to a baseline.

Changes to the project baseline can potentially lead to scope creep.
A project manager would make changes to the baseline iff it is necessary to keep the project on track, if the customer wants it and is willing to pay for it, or if it significantly improves the project.
A project manager would not allow changes to the baseline if it would lead to scope creep, or bring the project in another wrong direction.

\end{document}
